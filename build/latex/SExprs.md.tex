
\chapter{Defining Symbolic Expressions}
\section{Introduction}
You have been told that the Lambda Calculus is computationally universal, capable 
of expressing any algorithm. However, this most likely seems intangible. We will
begin to define a layer of abstraction over the Lambda Calculus which makes design
of programs begin to seem conceivable.

Our layer of abstraction will be a uniform language of Symbolic Expressions which
is a dialect of the language called Lisp. 
These symbolic expressions are parentheses enclosed arrays of symbols, taking on
different meanings based on their matching of patterns which we will define.

Here's an example:

\begin{figure}[htp]
\caption{}\label{scheme}
\begin{align*}
& (+ \; 2 \; 3)
\end{align*}
\end{figure}

The above evaluates to 5. In this case our expression is a function application
receiving two numbers as arguments. This syntax is very simple, uniform and
legible. Additionally, as you will see later on in this book, it is very easily
interpreted by program.

\section{Symbolic Expressions}
The language into which we are entering is one of symbolic expressions. All of our 
expressions will take the form defined by the following grammar. This uniformity
will make its definition in terms of Lambda Calculus far easier, and simplify its
later interpretation or compilation.

A Symbolic Expression

\begin{figure}[htp]
\caption{}\label{scheme}
\begin{align*}
& <expr> \; &::= \; <sexpr> \; | \; <atom>
\\& <sexpr> \; &::= \; (<seq>)
\\& <seq> \; &::= \; <dotted> \; | \; <list>
\\& <dotted> \; &::= \; <expr> \; | \; <expr> \; . \; <expr>
\\& <list> \; &::= \; <expr> \; | \; <expr> \; <exp>
\end{align*}
\end{figure}

\section{A Symbolic Language}
\subsection{Primitive Forms}
Now, these Symbolic Expressions or \emph{S-Expressions} can take any of a multitude of
forms. Of these, we will define meaning for forms of interesting patterns. We
begin, unsurprisingly, with an S-Expression which serves to create lambdas. All
forms matching the patterns which we discuss will be converted to the provided
form, labeled as the consequent.

\begin{figure}[htp]
\caption{}\label{scheme}
\begin{align*}
& (\text{lambda} \; (var) \; expr) \; &\implies \; \lambda var \; expr
\\& (\text{lambda} \; (var \; rest\dots) \; expr) \; &\implies \; \lambda var \; (\text{lambda} \; rest \; expr)
\end{align*}
\end{figure}

Essentially, we are saying that any expression of the form $(lambda args expr)$
should be a function of the provided arguments bearing the provided expression. 

Additionally, we provide a default case for our S-Expressions. Should no other
mentioned pattern be a match to a given expression, we will default to function
invocation. In other words, the following is a pattern we will match, with $fn$
being some foreign form not selected for elsewhere. 

\begin{figure}[htp]
\caption{}\label{scheme}
\begin{align*}
& (fn \; val) \; &= \; (fn)val
\\& (fn \; val \; rest\dots) \; &= \; ((fn)val \; rest)
\end{align*}
\end{figure}

The Lambda Calculus has now been fully implemented in our symbolic forms; however, 
we will add many more features for the sake of convenience. After all, our goal
was to add abstraction, not move a few symbols around!


\subsection{Evaluation of Symbolic Forms}
Before we continue, we'll look at some examples of our syntax as implemented so
far. In evaluating a Lambda Calculus expression, or in this case, derived form,
the single operation necessary is known as reduction. Reduction is conversion of
an expression of function application to a new expression, one derived by
substitution of the argument value.

To begin gaining familiarity with our language, we look at a function of two
variables. The function that follows performs $f$ on a value $x$, and then $f$
once more on the resultant value.

\begin{figure}[htp]
\caption{}\label{scheme}
\begin{align*}
& (\text{lambda} \; (f \; x) \; (f \; (f \; x)))
\end{align*}
\end{figure}

Now we will look at a similar form, a function of three variables. Below is a
function of values $g$, $f$, and $x$. The result is similar to that of the one
above, but this time replacing $(f x)$ with $(g f x)$.

\begin{figure}[htp]
\caption{}\label{scheme}
\begin{align*}
& (\text{lambda} \; (g \; f \; x) \; (f \; (g \; f \; x)))
\end{align*}
\end{figure}

So far we have covered some examples of Symbolic Expressions, but all of them have 
been of the lambda definition form. Furthermore, they have lacked any concrete
meaning. To explore the additional notation we have defined, we will apply the
former expression to the latter, which expands into the following.

\begin{figure}[htp]
\caption{}\label{scheme}
\begin{align*}
& ((\text{lambda} \; (g \; f \; x) \; (f \; (g \; f \; x))) \; (\text{lambda} \; (f \; x) \; (f \; (f \; x))))
\end{align*}
\end{figure}

This expression appears quite complex, so let's use the aforementioned reduction 
operation to simplify it. Recall from our definition of $lambda$ that a function
of multiple variables evaluated for one results in a function of one-less variable 
than the initial form. Hence we begin by substituting our argument for $g$
throughout the expression; later steps are of a similar nature.

\begin{figure}[htp]
\caption{}\label{scheme}
\begin{align*}
& &((\text{lambda} \; (g \; f \; x) \; (f \; (g \; f \; x))) \; (\text{lambda} \; (f \; x) \; (f \; (f \; x))))
\\& \implies \; &(\text{lambda} \; (f \; x) \; (f \; ((\text{lambda} \; (f \; x) \; (f \; (f \; x))) \; f \; x)))
\\& \implies \; &(\text{lambda} \; (f \; x) \; (f \; (f \; (f \; x))))
\end{align*}
\end{figure}

That looks much better! What we have arrived at is only slightly different than
our initial function of $f$ and $x$. The only change was the number of times that
$f$ was applied. We have explored the reduction of a symbolic expression to a
result; however, this result is far from tangible.

In order for computation by means of the Lambda Calculus to render a 
human-readable result, we will need a notation of expression exhibited by function
definitions. That means that, for example, the function 
$(lambda (a) (a (a (a a))))$ could serve to communicate the number four.

The expressive power of this notation is clear; however, application of a function 
to itself as in $(a a)$ is very rarely appropriate, and so we will slightly expand 
our expression of four to serve a more concrete purpose within the language. We
now expand our numeric function representing some number $n$ to accept two
operators, and return the application of the first parameter $n$ times to the
second parameter. Hence, the number four would look like the following.

\begin{figure}[htp]
\caption{}\label{scheme}
\begin{align*}
& (\text{lambda} \; (f \; n) \; (f \; (f \; (f \; (f \; n)))))
\end{align*}
\end{figure}

This notation, of course, translates just as well into an expression of any
number. The number three would look like the following.

\begin{figure}[htp]
\caption{}\label{scheme}
\begin{align*}
& (\text{lambda} \; (f \; n) \; (f \; (f \; (f \; n))))
\end{align*}
\end{figure}

Hopefully these examples have given you a feel for how this syntax can work, and
maybe even an early sense of how useful functions will emerge from the Lambda
Calculus.

\section{Foundations in Lambda Calculus}
To accompany our syntactic constructs, we will need to define some forms in the
Lambda Calculus, especially data-types and their manipulations. Our definitions
will be illustrated as equalities, like $id = \lambda xx$; however, syntactic
patterns will be expressed as implications. Recall that in Lambda Calculus there
are only functions, no literals or primitive data-types. To combat this apparent
shortcoming of the language, we will need to give data-types of interest a
functional form. The examples of the prior section were a preview into how our
conceptualization of numbers will behave.

\subsection{Numbers}
Numbers are a rather fundamental data-type, especially in modern computing.
Additionally, they are in most other languages seen as atomic and primitive.
However, we must provide a definition for the behavior of numbers in our language 
constituent of functions. We begin with a means of defining all natural numbers
inductively, via the successor.

Numbers
\begin{figure}[htp]
\caption{}\label{scheme}
\begin{align*}
& 0 \; &= \; \lambda \; f \; \lambda \; x \; x
\\& succ \; &= \; \lambda \; n \; \lambda \; f \; \lambda \; x \; (f)((n)f)x
\end{align*}
\end{figure}

Our definition of numbers is just like the examples from the previous section.
Notice that $1$, for example, could be easily defined as $1 = (succ 0)$, as could
any positive integer with enough applications of $succ$. Later on when we return
to syntactic features we will define all numbers in this way; the numbers will
take on their usual form as a string of decimal digits.

Numbers are our first data-type. Their definition is iterative in nature, with
zero meaning no applications of the function $f$ to $x$. We now will define some
elementary manipulations of this data-type, i.e., basic arithmetic. The following
definitions are pretty straightforward; nearly all of them consist exclusively of
iterative application of a more primitive function to a base value.

Arithmetic
\begin{figure}[htp]
\caption{}\label{scheme}
\begin{align*}
& + \; &= \; \lambda \; n \; \lambda \; m \; ((n)succ)m
\\& * \; &= \; \lambda \; n \; \lambda \; m \; ((n)(sum)m)0
\\& pred \; &= \; \lambda \; n \; \lambda \; f \; \lambda \; z \; ((((n) \; \lambda \; g \; \lambda \; h \; (h)(g)f)\lambda \; u \; z)\lambda \; u \; u)
\\& - \; &= \; \lambda \; n \; \lambda \; m \; ((m)pred)n
\end{align*}
\end{figure}

Addition merely takes advantage of the iterative nature of our numbers to apply
the successor $n$ times, starting with $m$. In a similar manner, multiplication
applies addition repeatedly starting with zero. The predecessor is much more
complicated, so let's work our way through its evaluation.

We'll begin our exploration of the $pred$ function by looking at the value of two. 
Since two equals $(succ)(succ)0$ we can work out its Lambda form, or simply take
as a given that is the following.

\begin{figure}[htp]
\caption{}\label{scheme}
\begin{align*}
& 2 \; = \; \lambda \; f \; \lambda \; x \; (f)(f)x
\end{align*}
\end{figure}

Now we can evaluate $pred$ for this value. $pred$ has been defined already, but
let's briefly render it in the following more succinct form. The form below is
very easily translated back to Lambda Calculus and should serve to cut through at
least a portion of the complexity of the definition.

\begin{figure}[htp]
\caption{}\label{scheme}
\begin{align*}
& pred \; = \; \lambda \; n \; \lambda \; f \; \lambda \; z \; ((\lambda \; g \; \lambda \; h \; (h)(g)f)^{n} \; \lambda \; u \; z) \; \lambda \; u \; u
\end{align*}
\end{figure}

With the above rendering of the definition in mind, we aim to reduce an
application of $pred$ to $2$ to a result.

\begin{figure}[htp]
\caption{}\label{scheme}
\begin{align*}
& (\lambda \; n \; \lambda \; f \; \lambda \; z \; ((\lambda \; g \; \lambda \; h \; (h)(g)f)^{n} \; \lambda \; u \; z) \; \lambda \; u \; u) \; 2
\\& (\lambda \; f \; \lambda \; z \; ((\lambda \; g \; \lambda \; h \; (h)(g)f)^2 \; \lambda \; u \; z) \; \lambda \; u \; u)
\end{align*}
\end{figure}
Now that we have reduced the expression to the form of our prior rendering of
$pred$, we will expand it into a true Lambda Calculus form and continue our
reduction.

\begin{figure}[htp]
\caption{}\label{scheme}
\begin{align*}
& (\lambda \; f \; \lambda \; z \; (((\lambda \; g \; \lambda \; h \; (h)(g)f) \; (\lambda \; g \; \lambda \; h \; (h)(g)f)) \; \lambda \; u \; z) \; \lambda \; u \; u)
\end{align*}
\end{figure}

In the following conversions, as in all, our reductions will need to take place in 
a right-to-left direction when evaluating expressions of the form $(f)(g)x$.
Recall that our goal here is to reduce a complex form to simplistic result, and we 
have already made significant progress.

\begin{figure}[htp]
\caption{}\label{scheme}
\begin{align*}
& (\lambda \; f \; \lambda \; z \; ((\lambda \; g \; \lambda \; h \; (h)(g)f) \; (\lambda \; h \; (h)(\lambda \; u \; z)f)) \; \lambda \; u \; u)
\\& (\lambda \; f \; \lambda \; z \; ((\lambda \; g \; \lambda \; h \; (h)(g)f) \; (\lambda \; h \; (h)z)) \; \lambda \; u \; u)
\\& (\lambda \; f \; \lambda \; z \; (\lambda \; h \; (h)(\lambda \; h \; (h)z)f) \; \lambda \; u \; u)
\\& (\lambda \; f \; \lambda \; z \; (\lambda \; h \; (h)(f)z) \; \lambda \; u \; u)
\end{align*}
\end{figure}
The above were all mere substitutions, as should be expected. If any were unclear, 
try working those steps out in a notebook. We are now finally ready to reduce the
application of the identity ($lambda u u$) and achieve our final result.

\begin{figure}[htp]
\caption{}\label{scheme}
\begin{align*}
& \lambda \; f \; \lambda \; z \; (\lambda \; u \; u)(f)z
\\& \lambda \; f \; \lambda \; z \; (f)z
\end{align*}
\end{figure}

Our result was a single application of $f$ to $z$, i.e., one. Hence you have seen 
that at least in this case, the $pred$ function did its job. Achieving an
intuitive grasp of how it works is unfortunately not as straight-forward. If you
wish to, keep in mind that $lambda u z$ maps a value to the numeric starting
point, and $lambda u u$ leaves an expression alone. So the decrement occurs by
the setting of the origin later than it would normally occur.

With our complex definition of the predecessor complete, subtraction is trivial.
Once again we perform an iterative process on a base value, this time that process 
is $pred$.

\subsection{Booleans}
Having defined numbers and their manipulations, we will work on booleans. Booleans 
are the values of true and false, or in our syntax, $t$ and $f$. Booleans are
quite necessary in expressing conditional statements; thus the concomitant $if$
function. These values will give us great power in their ability to branch results 
to a function, in a sense constructing piece-wise functions. It is by this ability 
that we are able to form a multitude of inductive definitions, as well as other 
important forms.

Booleans
\begin{figure}[htp]
\caption{}\label{scheme}
\begin{align*}
& \#t \; &= \; \lambda \; a \; \lambda \; b \; (a)id
\\& \#f \; &= \; \lambda \; a \; \lambda \; b \; (b)id
\\& if \; &= \; \lambda \; p \; \lambda \; t \; \lambda \; f \; ((p)\lambda \; \_ \; t)\lambda \; \_ \; f
\end{align*}
\end{figure}

The key to our booleans is that they accept two functions as parameters, functions 
that serve to encapsulate values, of which one will be chosen. Once chosen, that 
function is executed with the arbitrarily-chosen identity as an argument. This
method of wrapping the decision serves as a means of lazy evaluation, and is fully 
realized in the lambda-underscores wrapping the branches of an $if$ statement.

Now, since of course no boolean system is complete without some boolean algebra, 
we define $and$ and $or$. These functions perform the operations you would expect; 
$(and a b)$ is true only when both $a$ and $b$ are true, but $(or a b)$ is true if 
either argument is true. Their definitions follow easily from our $if$ function. 
Keep in mind that both of these functions operate only on booleans.

Boolean Algebra
\begin{figure}[htp]
\caption{}\label{scheme}
\begin{align*}
& and \; &= \; \lambda \; a \; \lambda \; b \; (((if)a)b)\#f
\\& or \; &= \; \lambda \; a \; \lambda \; b \; (((if)a)\#t)b
\end{align*}
\end{figure}

With boolean manipulation and conditionals in hand, we need some useful predicates 
to utilize them. We define some basic predicates on numbers with the following. 
$eq$ will be very useful in later developments; it is one of McCarthy's elementary 
functions.

Numerical Predicates
\begin{figure}[htp]
\caption{}\label{scheme}
\begin{align*}
& zero? \; &= \; \lambda \; n \; ((n)λx\#f)\#t
\\& leq \; &= \; \lambda \; a \; \lambda \; b \; (zero?)((-)m)n
\\& eq \; &= \; \lambda \; a \; \lambda \; b \; (and \; (leq \; a \; b) \; (leq \; b \; a))
\end{align*}
\end{figure}

The above predicates serve to identify traits of a given number or given numbers. 
$zero?$ is true when a number is zero, $leq$ is true when the first number is less 
than or equal to the second, and $eq$ determines whether two numbers are equal.

\subsection{Pairs}
Finally we reach the most important part of our S-Expressions, their underlying 
lists. That is to say, every Symbolic Expression is innately a list of other 
expressions, whether atomic or symbolic, and these lists serve as an analog to 
that data-type. To construct lists we will opt for a sort of linked-list 
implementation in our lambda definitions. We begin with a pair and a $nil$ 
definition, each readily revealing their type by opting for either the passed $c$ 
or $n$ function.

Pairs
\begin{figure}[htp]
\caption{}\label{scheme}
\begin{align*}
& cons \; &= \; \lambda \; a \; \lambda \; b \; \lambda \; c \; \lambda \; n \; ((c)a)b
\\& nil \; &= \; \lambda \; c \; \lambda \; n \; (n)id
\end{align*}
\end{figure}

$cons$ constructs a pair when given two values, and accepts a function which will 
receive the two items to manipulate. $nil$ on the other hand serves as a sort of 
empty pair, and instead fires the second provided function to identify itself as 
such.

Now, once again we follow a defined data-type with its manipulations. Just as did 
McCarthy, we will provide $car$ and $cdr$ as additional elementary functions, with 
$pair?$ and $null?$ serving as complements to each other in determining the end of 
a list. $car$ returns the first value of a pair, and $cdr$ the second. Their names 
are quite historical and refer to address access of a pair in memory, but you can 
just think of them as $/k \alpha r/$ and $/k \mho d e r/$.

Pair Operations
\begin{figure}[htp]
\caption{}\label{scheme}
\begin{align*}
& car \; &= \; \lambda \; l \; (((l)\lambda \; a \; \lambda \; b \; a)id)
\\& cdr \; &= \; \lambda \; l \; (((l)\lambda \; a \; \lambda \; b \; b)id)
\\& pair? \; &= \; \lambda \; l \; (((l)\lambda \; \_ \; \lambda \; \_ \; \#t)\lambda \; \_ \; \#f)
\\& null? \; &= \; \lambda \; l \; (((l)\lambda \; \_ \; \lambda \; \_ \; \#f)\lambda \; \_ \; \#t)
\end{align*}
\end{figure}

$car$ provides that pair with a pair handling function that returns the first 
element, and an arbitrary $nil$ handling function. Similarly, $cdr$ provides a 
pair handling function returning the second element. $pair?$ and $nil?$ are 
logical opposites to each other, each provides a pair- and nil-handling function, 
returning either $t$ or $f$ as is appropriate.

Together, these functions are sufficient for designing a list implementation. The 
implementation that comes naturally is known as a linked-list. A linked-list is 
essentially either a pair of an element and a linked-list or $nil$. If that is 
unclear, think of a tree with a fractal structure. The tree consists of a leaf and 
a child tree, which in turn has both leaf and child tree, until the tree ends with 
$nil$ for a child tree.

\subsection{Recursion}
Our last definition will be a bit more esoteric, or at least complex. We define a 
\emph{Y Combinator}. This function, $Y$, will allow another to be executed accepting 
itself as an argument.

Self-Reference by a Combinator
\begin{figure}[htp]
\caption{}\label{scheme}
\begin{align*}
& Y \; = \; λf(λx(f)(x)x)λx(f)(x)x
\end{align*}
\end{figure}

You need not delve into its innerworkings; rather, let's work through an example. 
Let's say you want to define a function that will evaluate the factorial of a 
number $n$. Well then the fundamental idea would be to do something like the 
following.

\begin{figure}[htp]
\caption{}\label{scheme}
\begin{align*}
& fact \; = \; (\text{lambda} \; (n) \; (* \; n \; \dots))
\end{align*}
\end{figure}

Well the question remains, what should be present instead of the dots? Nothing we 
have discussed would give any help in answering that, besides the Y Combinator. If 
you recall from Mathematics, the factorial has an inductive definition like the 
following.

Inductive Definition of Factorial
\begin{figure}[htp]
\caption{}\label{scheme}
\begin{align*}
& 0! \; &= \; 1
\\& n! \; &= \; n \; * \; (\text{n-1)!}
\end{align*}
\end{figure}

Our task is to translate this into our Symbolic Language; however, we wish to 
generalize this idea a bit, not to add a special expression type for every 
inductive definition we think of. Hence our duty is to make a factorial function 
which is self-aware, if you will. The following is a rough form of the concept.

\begin{figure}[htp]
\caption{}\label{scheme}
\begin{align*}
& fact \; = \; (\text{lambda} \; (fact \; n) \; (* \; n \; (fact \; (pred \; n))))
\end{align*}
\end{figure}

There remains one issue! We have not handled the base case present in our prior 
definition. To achieve this in the Lambda Calculus we will utilize an if 
statement; this use case was alluded to earlier.

\begin{figure}[htp]
\caption{}\label{scheme}
\begin{align*}
& fact \; = \; (\text{lambda} \; (fact \; n) \; (if \; (zero? \; n) \; (succ \; 0) \; (* \; n \; (fact \; (pred \; n))))
\end{align*}
\end{figure}

This is a complete realization of the definition, but there remains one problem. 
How are we to pass $fact$ the value of $fact$? This is when the Y Combinator comes 
in. The invocation of $(Y fact)$ will form a factorial function, aware of itself 
for the sake of recursion, accepting the single variable $n$.

\subsection{Conclusion}
We have now laid a good foundation upon which our Symbolic Expressions can exist. 
As should be expected, lists will be our primary data-structure in our language of 
S-Expressions.

\section{Special Forms of S-Expressions}
\subsection{Numbers}
Returning to our prior definition of numbers, we will now define arbitrarily long 
strings of decimal digits. As you can see, the following defines numbers by either 
matching single digits and defining them as a successor, or by matching leading 
digits and a final digit and evaluating them separately.

\begin{figure}[htp]
\caption{}\label{scheme}
\begin{align*}
& 1 \; &= \; (succ \; 0)
\\& 2 \; &= \; (succ \; 1)
\\& \dots
\\& 9 \; &= \; (succ \; 8)
\\& ten \; &= \; (succ \; 9)
\\& d\dots0 \; &= \; (mul \; d\dots \; ten)
\\& d\dots1 \; &= \; (sum \; (mul \; d\dots \; ten) \; 1)
\\& \dots
\\& d\dots9 \; &= \; (sum \; (mul \; d\dots \; ten) \; 9)
\end{align*}
\end{figure}

Hopefully the above is a clear embodiment of our decimal number system. Place 
value is achieved by two measures, (a) inductive definition, and (b) 
multiplication by ten. This pattern of recursive definition and symbolic pattern 
matching will be at the heart of our language constructs.

\subsection{A Predicate for Atoms}
In order that functions may have a sense for the value with which they have been 
presented, we provide a predicate for determining whether a given value is atomic.

\begin{figure}[htp]
\caption{}\label{scheme}
\begin{align*}
& (atom? \; (a \; b\dots)) \; &\implies \; \#f
\\& (atom? \; a) \; &\implies \; \#t
\end{align*}
\end{figure}

\subsection{List Literals}
We define our lists inductively based on the pair-constructing $cons$ function we 
defined earlier. We choose to name this function $quote$ because it is treating 
the entire expression as a literal, rather than as a symbolic expression. More 
importantly, the syntax of passed lists is indistinguishable from a regular 
S-Expression, hence we are utilizing the \emph{quoted} form of such an expression.

The following definition has a rather sensitive notation. Quotes show that an 
atomic value, that is, a value referred to in our grammar as $<atom>$ or more 
importantly, a value referred to as $<var>$ in our grammar of the Lambda Calculus, 
is being matched. This is unique from most cases in which a portion of a pattern 
is being labeled by a variable. Additionally, the italicized \emph{ab...} is meant to 
label the first letter and rest of a string as $a$ and $b$, respectively.

\begin{figure}[htp]
\caption{}\label{scheme}
\begin{align*}
& (quote \; (a)) \; &\implies \; cons \; a \; nil
\\& (quote \; (a \; rest\dots)) \; &\implies \; (cons \; (quote \; a) \; (quote \; (rest\dots))
\\& (quote \; a \; rest\dots) \; &\implies \; (cons \; (quote \; a) \; (quote \; (rest\dots)))
\\& (quote \; "0") \; &\implies \; 0
\\& \dots
\\& (quote \; "99") \; &\implies \; 99
\\& \dots
\\& (quote \; "ab\dots") \; &\implies \; (cons \; a \; (quote \; b\dots))
\\& (quote \; "a") \; &\implies \; (cons \; 97 \; nil)
\\& \dots
\\& (quote \; "z") \; &\implies \; (cons \; 122 \; nil)
\end{align*}
\end{figure}

In addition to defining the $quote$ function, we will provide a shorthand for the 
operation. So often will we need to define list literals that it makes perfect 
sense for us to make it as brief as possible.

\begin{figure}[htp]
\caption{}\label{scheme}
\begin{align*}
& 'a \; \implies \; (quote \; a)
\end{align*}
\end{figure}

Quoted forms will come up often in writing list literals, atomic values derived 
from strings, i.e., \emph{atoms}, and forming more complex data-structures from lists 
such as tables.

\subsection{Equality}
We have built up an array of atomic values, and a way of keeping them literal. Now 
we need a way of recognizing them, by means of equivalence. $eq$ already solves 
this problem for numbers, but not for other quoted atoms. We generalize $eq$ to 
all expressions in our definition of $equal?$.

\begin{figure}[htp]
\caption{}\label{scheme}
\begin{align*}
& (equal? \; (a \; b\dots) \; (c \; d\dots)) \; &\implies \; (and \; (equal? \; a \; c) \; (equal? \; (b\dots) \; (d\dots)))
\\& (equal? \; a \; b) \; &\implies \; (eq \; a \; b)
\end{align*}
\end{figure}

The above definition is inductive in nature. It provides a base case in which 
equivalence is determined by the Lambda Calculus definition of $eq$, and an 
inductive step in which lists are equivalent only if their constituents are equal.

\subsection{Variable Definition}
Now we add some \emph{syntactic sugar} that will make it easier to store values that 
will be used in an expression. $let$ and $let*$ set a single value and a list of 
values, respectively, to be utilized in a given expression. $letrec$ takes this 
idea in another direction, performing the Y-Combinator on a passed function to 
prepare it for recursion in the passed expression.

\begin{figure}[htp]
\caption{}\label{scheme}
\begin{align*}
& (\text{let} \; var \; val \; expr) \; &\implies \; ((\text{lambda} \; (var) \; expr) \; val)
\\& (let* \; ((var \; val)) \; expr) \; &\implies \; (\text{let} \; var \; val \; expr)
\\& (let* \; ((var \; val) \; rest\dots) \; expr) \; &\implies \; ((\text{lambda} \; (var) \; (let* \; (rest\dots) \; expr)) \; val)
\\& (\text{letrec} \; var \; fn \; expr) \; &\implies \; (\text{let} \; var \; (Y \; (\text{lambda} \; f \; fn)) \; expr)
\end{align*}
\end{figure}

Once again we provide an inductive definition, and here we finally utilize the Y 
Combinator we discussed with regard to recursive functions.

\subsection{Conclusion}
We have formed a basic language consisting of Symbolic Expressions defined by the 
Lambda Calculus. All of our expressions are reducible to Lambda forms, yet clearer 
or more concise given their symbolic form. This language will be utilized for 
expression of all sorts of computational ideas, including algorithms, simulators, 
and interpreters. Our choice of syntax was purely aesthetic; Lambda Calculus is 
sufficient for communication with machine, however, our language of Symbolic 
Expressions is far friendlier to a human reader. This motivation reveals the 
additional motivation for our construction of this language, to form a clear, 
formal, extensible, and uniform means of communicating ideas.
