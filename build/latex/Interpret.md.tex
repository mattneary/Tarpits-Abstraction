
\chapter{Mechanical Interpretation of a Language}
The most important revelation in learning the art of programming is that the 
language in which you work is completely arbitrary. More specifically, the 
language in which you express concepts was defined in terms of another language 
at some point. We have already made clear this concept in our definition of our 
symbolic language. We now turn to the interpretation of one language within 
another.

\section{Lambda Calculus}
The language which we will interpret is one with which we are already familiar, 
Lambda Calculus. The Lambda Calculus has very simple syntax and will thus not 
be too hard to interpret. Recall the syntax, which is composed of the following 
expressions.

\begin{itemize}
  \item A variable reference.
  \item A function definition of the form $\lambda a b$ where $a$ is a variable reference and $b$ is an expression.
  \item A function application of the form $(a)b$ where $a$ is an expression, as is $b$.
\end{itemize}

Note that the generality of the third form, function application, is what gives 
this syntax its description as the Untyped Lambda Calculus. Since no 
qualification is given to the expression which will be passed argument, this 
language is without types.

\subsection{Lambda Calculus in S-Expressions}
In expression the Lambda Calculus in S-Expressions, we will utilize the $quote$ 
function as well as the structure inherent of parenthetical expressions in these 
expressions. Hence an example of an expression which could be evaluated is the
one presented in Figure~\ref{fig:sexprLambda}.

\begin{figure}[htp]
\footnotesize
\caption{}\label{fig:sexprLambda}
\begin{align*}
& '(lam \; x \; lam \; y \; (x) \; y)
\end{align*}
\end{figure}

\subsection{An Evaluator}
We define our evaluator pretty easily. Note that we will begin by defining an 
$apply$ function. This function accepts a function and list of arguments, and 
then applies each of these arguments to a lambda one by one.

\begin{figure}[htp]
\footnotesize
\caption{}\label{fig:applySetDef}
\begin{align*}
& (define \; (\text{apply-set} \; fn \; args)
\\& \quad (if
\\& \qquad (null? \; args)
\\& \qquad fn
\\& \qquad (\text{apply-set} \; (fn \; (car \; args)) \; (cdr \; args))))
\end{align*}
\end{figure}

Just like our definition of the syntax, our evaluator handles variable reference, 
lambda definition, and function application. 

Variable reference is a problem very easily solved. Atoms are considered 
variable references, and hence serve as keys in accessing values from the 
environment hash.

Function definition is achieved by returning a lambda of a single variable for 
expressions of the necessary form. Within the lambda, the passed argument is 
appended to the environment with the argument name as its key. The function body 
is then evaluated.

Function application is the default case, thus we match against the antecedent 
$t$. The applier $\text{apply-set}$ is then called with the evaluated form of the first 
argument and the evaluated arguments. This architecture is an explicit choice 
\emph{not} to opt for a lazy method of evaluation.

\begin{figure}[htp]
\footnotesize
\caption{}\label{fig:evallamDef}
\begin{align*}
& (evallam \; (lambda \; (evallam \; expr \; env)
\\& \quad (cond \; (((atom? \; expr) \; (assoc \; expr \; env))
\\& \qquad \qquad \; ((equal? \; (car \; expr) \; 'lam) \; 
\\& \qquad \qquad \quad \; (lambda \; (x) \; 
\\& \qquad \qquad \qquad \; (evallam \; 
\\& \qquad \qquad \qquad \quad \; (cddr \; expr) \; 
\\& \qquad \qquad \qquad \quad \; (set \; (cadr \; expr) \; x \; env))))
\\& \qquad \qquad \; ((null? \; (cdr \; expr)) \; 
\\& \qquad \qquad \quad (evallam \; (car \; expr) \; env))
\\& \qquad \qquad \; (\#t \; 
\\& \qquad \qquad \quad \; (\text{apply-set} \; 
\\& \qquad \qquad \qquad \; (evallam \; (car \; expr) \; env) \; 
\\& \qquad \qquad \qquad \; (map \; 
\\& \qquad \qquad \qquad \quad \; (lambda \; (expr) \; (evallam \; expr \; env)) \; 
\\& \qquad \qquad \qquad \quad \; (cdr \; expr))))))))
\end{align*}
\end{figure}

\subsection{Evaluation of Forms}
An example of a form which could be evaluated is presented in Figure~\ref{fig:evalFormExample}. 

\begin{figure}[htp]
\footnotesize
\caption{}\label{fig:evalFormExample}
\begin{align*}
& '(lam \; x \; lam \; y \; ((x) \; y) \; 1)
\end{align*}
\end{figure}

\section{Flat-Input Lambda Calculus }
In the prior implementation of an interpreter, we took advantage of the
structure inherent to a nested S-Expression. This approach was sufficient for
our initial purposes; however, to separate our interpreter from the details of
its use within our Symbolic Language, we will now allow its interpretation to
apply to a flat list of atoms.  In order to represent the expression previously
expressed by nested S-Expressions, we will now utilize some symbols which will
represent parenthetical expressions.  The expression in
Figure~\ref{fig:flatInputExample} is an example of this new flat structure.

\begin{figure}[htp]
\footnotesize
\caption{}\label{fig:flatInputExample}
\begin{align*}
& '(lam \; x \; lam \; y \; < \; x \; > \; y)
\end{align*}
\end{figure}

Of course, this use of $<>$ symbols would extend to any instance of parentheses 
in our prior method.

With our new, less inherently structured approach, we will need to provide an 
additional layer of parsing. Parsing will provide this missing aspect of structure. 
Parsing parentheses is actually our most complex algorithm yet attempted. We will 
take this algorithm's implementation as an opportunity to experiment with the 
second style of programming we have yet to investigate, imperative programming.

\subsection{The Two Styles of Programming}
There are two basic approaches to programming, derived from the two original 
theories of computation. We have talked far more about functional programming 
tactics in prior sections of this book, leaving imperative programming on the 
sidelines. However, the problem at hand is a great case study in the relation 
between imperative and functional languages. We will begin with an imperative 
implementation, and then port the code over to our current language of choice.

\subsection{Imperative Constructs}
In our exploration of imperative programming, we will encounter a few new operators, 
and utilize some new idioms. We will provide a purely functional, that is, without
mutation, implementation of these constructs as well. In later chapters, we will be
able to automate the utilization of these analogs identified.

\subsection{Mutators}
The main difference between imperative and \emph{purely} functional programming is
the presence of mutability. In functional programs, a value can be defined but
not mutated; however, when taking the imperative approach, values will often be
set to a new value after their definition. The code in
Figure~\ref{fig:defineSetExample} is an example of this behavior.

\begin{figure}[htp]
\footnotesize
\caption{}\label{fig:defineSetExample}
\begin{align*}
& (define \; x \; 5)
\\& (set! \; x \; (* \; 2 \; x))
\\& \implies \; x \; = \; 10
\end{align*}
\end{figure}

The $define$ operator serves to allocate a variable and initiate it with a value. 
This variable, $x$, can then be accessed throughout the procedure, and even mutated 
to equal a new value. In the example, it was initiated as 5, but $set!$ to 10.

In order to simulate mutation, we will need a means of manipulating an environment
accessed and mutated by a myriad of expressions. The pure means of achieving this,
as we have previously discussed, is to call a function with a mutated duplicate of
the environment. In this case, that function would be defined to use recursion.

Our recursive function will accept an environment and an expression address, i.e.,
index, as argument, and return either the result or a recursion with a mutated
environment and at a different expression address. This contraption is similar to 
a register machine in many ways; an analog which we will further analyze in later
sections. What follows is an implementation using these concepts of the prior
imperative procedure.

\begin{figure}[htp]
\footnotesize
\caption{}\label{fig:recursiveImper}
\begin{align*}
& (letrec \; 
\\& \quad main \; 
\\& \quad (lambda \; (main \; env \; start)
\\& \qquad (get \; (list \; (main \; (set \; 'x \; 5 \; env) \; (+ \; 1 \; start))
\\& \qquad \qquad \qquad \quad \; (main \; (set \; 'x \; (* \; 2 \; (assoc \; 'x \; env)) \; env) \; (+ \; 1 \; start))
\\& \qquad \qquad \qquad \quad \; (equal? \; (assoc \; 'x \; env) \; 5)) \; start))
\\& \quad \dots)
\end{align*}
\end{figure}

In Figure~\ref{fig:recursiveImper}, note that we omitted the second $equal?$,
because only one of them bore an actual effect. We now move on to address more
complex issues of this impure approach.

In programming languages, \emph{scope} refers to the region over which a variable is
accessible. The scoping of a variable is specified by the define operator; hence the 
code in Figure~\ref{fig:setScopeExample} is another example of this behavior.

\begin{figure}[htp]
\footnotesize
\caption{}\label{fig:setScopeExample}
\begin{align*}
& (define \; x \; 5)
\\& ((lambda \; (y)
\\& \quad (set! \; x \; y)) \; 12)
\\& (equal? \; x \; 12) \; 
\\& ;; \; \#t
\end{align*}
\end{figure}

Of note is the fact that the $define$ occurred separate from any function. This 
means that the defined variable will now take on the \emph{global} scope, being accessible 
and mutable from within any function. 

In translating the definition and application of the lambda to a purely
functional procedure, we will provide the action of the lambda as a prelude to
the rest of the procedure. The invocation of the lambda will require that we
set the index to which the flow of control should return after completion of
the lambda. This takes the form of a variable $ret$ defined on the environment.
All other methods in Figure~\ref{fig:imperFunc} are similar to those in prior
procedures.

\begin{figure}[htp]
\footnotesize
\caption{}\label{fig:imperFunc}
\begin{align*}
& (letrec \; 
\\& \quad main \; 
\\& \quad (lambda \; (main \; env \; start)
\\& \qquad (get \; (list \; (main \; (set \; 'x \; (assoc \; 'y \; env) \; env) \; (assoc \; 'ret \; env))
\\& \qquad \qquad \qquad \quad \; (main \; (set \; 'x \; 5 \; env) \; (+ \; 1 \; start))
\\& \qquad \qquad \qquad \quad \; (main \; (set \; 'ret \; 3 \; (set \; 'y \; 12 \; env)) \; 0)
\\& \qquad \qquad \qquad \quad \; (equal? \; (assoc \; 'x \; env) \; 5)) \; start))
\\& \quad \dots)
\end{align*}
\end{figure}

In the Figure~\ref{fig:imperFunc}, our starting index would instead be 1, in
order to begin at the first line of the imperative program and avoid the
definition of the lambda used later on in the procedure.

If the $define$ of the prior example had instead occurred within a function
definition, as in Figure~\ref{fig:strictScopeExample}, it would only be
accessible from within that function, or other functions defined within it.

\begin{figure}[htp]
\footnotesize
\caption{}\label{fig:strictScopeExample}
\begin{align*}
& (define \; scope \; (lambda \; (x)
\\& \quad (define \; y \; x)))
\\& (scope \; 5)
\\& y
\\& ;; \; The \; written \; variable, \; y, \; will \; be \; inaccessible.
\end{align*}
\end{figure}

In Figure~\ref{fig:strictScopeExample} we demonstrate definition with a
single-function scope. Thus the $define$ is fulfilling the same role as $let$
did in prior programs. However, since $define$ does not accept an expression
which it will govern, the example definition is of no effect. In the following
section we display a means of making use of this sort of $define$ statement.

To simulate this, we would need to add a sort of inner scope to our function calls, 
exhibited in the form of jumping to another instruction. We will, for the sake of
simplicity, create an inner environment, known as a \emph{closure}, as a value on the
outer, or normal, environment. Then, prior to returning, we will clear the inner 
environment by setting it to $nil$.

\begin{figure}[htp]
\footnotesize
\caption{}\label{fig:closureScope}
\begin{align*}
& (letrec \; 
\\& \quad main \; 
\\& \quad (lambda \; (main \; env \; start)
\\& \qquad (get \; 
\\& \qquad \quad (list \; 
\\& \qquad \qquad (let* \; ((outer \; env)
\\& \qquad \qquad \qquad \quad \; (inner \; 
\\& \qquad \qquad \qquad \qquad \; (set \; 'y \; 
\\& \qquad \qquad \qquad \qquad \quad \; (assoc \; 'x \; (assoc \; 'inner \; outer)) \; 
\\& \qquad \qquad \qquad \qquad \quad \; (assoc \; 'inner \; outer))
\\& \qquad \qquad \qquad \quad \; (outer \; (set \; 'inner \; inner \; outer)))
\\& \qquad \qquad \qquad \quad \; (main \; outer \; (+ \; 1 \; start)))
\\& \qquad \qquad \qquad \quad (main \; (set \; 'inner \; nil \; outer) \; (assoc \; 'ret \; outer))
\\& \qquad \qquad \qquad \quad (main \; 
\\& \qquad \qquad \qquad \qquad (set \; 
\\& \qquad \qquad \qquad \qquad \quad 'inner \; 
\\& \qquad \qquad \qquad \qquad \quad (set \; 'x \; 5 \; (assoc \; 'inner \; env)) \; 
\\& \qquad \qquad \qquad \qquad \quad env) \; 
\\& \qquad \qquad \qquad \qquad (+ \; 1 \; start))
\\& \qquad \qquad \qquad \quad (main \; 
\\& \qquad \qquad \qquad (set \; 'ret \; 4 \; (set \; 'y \; 12 \; env)) \; 
\\& \qquad \qquad \qquad \qquad 0)
\\& \qquad \qquad \quad (assoc \; 'y \; env)) \; 
\\& \qquad start))
\\& \quad \dots)
\end{align*}
\end{figure}

Of note is the fact that the $define$ from within a closure translated into a $set$
upon the inner environment. If we were aiming to automate this process, we would
instead maintain an image of the original environment, and simply revert to that 
image after execution of the function.

\subsection{Multiple Expression Procedures}
In our earlier, purely-functional programs, a procedure consisting of multiple
expressions would have been no use. Without side-effects, only the final
expression could bear any form of result. However, now investigating an
imperative approach, a procedure may utilize multiple expressions, each
contributing its own mutation to a final effect.
Figure~\ref{fig:multiExprExample} is an example of this in practice; the syntax
is simply a chain of expressions where an individual would have previously
existed.

\begin{figure}[htp]
\footnotesize
\caption{}\label{fig:multiExprExample}
\begin{align*}
& (define \; incr \; (lambda \; (x)
\\& \quad (define \; y \; (+ \; x \; 1))
\\& \quad y))
\end{align*}
\end{figure}

Obviously, this example is of no utility. The desired function could be just as 
easily achieved with a single expression. Useful examples, however, will present 
themselves in the following sections.

\subsection{Loop Constructs}
You will often see imperative programming avoiding use of recursion. Rather, these 
programs will often iterate, mutating the environment in each step. For convenience 
in utilization of this approach, we define a function for constructing a range over 
which to iterate.

\begin{figure}[htp]
\footnotesize
\caption{}\label{fig:rangeLoopDef}
\begin{align*}
& (define \; range \; (lambda \; (x)
\\& \quad (if \; (equal? \; x \; 0)
\\& \qquad nil
\\& \qquad (cons \; (- \; x \; 1) \; (range \; (- \; x \; 1))))))
\end{align*}
\end{figure}

The definition in Figure~\ref{fig:rangeLoopDef} is pretty straight-forward,
much like earlier function definitions. Note that the ranges are of the form 0,
1, ..., n-1. Here's an example of this function being used to calculate a
factorial.

\begin{figure}[htp]
\footnotesize
\caption{}\label{fig:imperFactExample}
\begin{align*}
& (define \; fact \; (lambda \; (x)
\\& \quad (define \; ans \; 1)
\\& \quad (map \; (range \; x) \; (lambda \; (n)
\\& \qquad (set! \; ans \; (* \; and \; (+ \; 1 \; n)))))
\\& \quad ans))
\end{align*}
\end{figure}

The starting value of the answer is 1, just like the sort of inductive definitions we 
provided earlier in the book. The final answer is then achieved by repeated 
multiplication performed on the previous $ans$. In the case of 5, for example, the 
accumulator $ans$ takes on the values presented in Figure~\ref{fig:ansValues}.

\begin{figure}[htp]
\footnotesize
\caption{}\label{fig:ansValues}
\begin{align*}
& 1
\\& \implies \; 1*1 \; \implies \; 1
\\& \implies \; 1*2 \; \implies \; 2
\\& \implies \; 2*3 \; \implies \; 6
\\& \implies \; 4*6 \; \implies \; 24
\\& \implies \; 5*24 \; \implies \; 120
\end{align*}
\end{figure}

\subsection{An Imperative Solution}
Now we will jump right in to the non-trivial problem at hand, restated below.

"Given a string of nested angle-bracket delimited groups, return a
nested list containing the contents of these groups. For example,
given the list of characters $'(a < b c > d)$ return $'(a (b c) d)$."

Since we are taking an imperative approach, think, "What is the easily defined iterative 
process underlying this problem?" The answer is clearly navigation of the string, and so 
we begin with a $range$-based loop that will cycle through each character of the string 
in order.

\begin{figure}[htp]
\footnotesize
\caption{}\label{fig:loopParse}
\begin{align*}
& (define \; parse \; (lambda \; (expr) \; 
\\& \quad (map \; (range \; (length \; expr)) \; (lambda \; (i)
\\& \qquad (define \; read \; (get \; expr \; i))
\\& \qquad // \; \dots
\\& \quad )))
\end{align*}
\end{figure}

Now we will need to describe a slightly more specific strategy in performing the desired 
process.

\begin{itemize}
  \item A parenthetical will be split from the string, with a segment, although possibly an empty one, before and after it.
  \item Once a parenthetical has been removed, we will need to recurse on these segments, i.e., the parenthetical and the portion after it.
\end{itemize}

To make our way toward this implementation, we will define a variable $before$ that will 
hold the segment of the string occurring prior to any parenthetical; a variable $accum$ 
that will hold characters that have been read in but whose destination has yet to be 
determined, in this way serving as a cache; $paren$ which will hold a separated out 
parenthetical; and $found$ which will be true if and only if a parenthetical has been 
parsed.

\begin{figure}[htp]
\footnotesize
\caption{}\label{fig:definsPrelude}
\begin{align*}
& (define \; parse \; (lambda \; (expr) \; 
\\& \quad (define \; before)
\\& \quad (define \; accum \; nil)
\\& \quad (define \; paren)
\\& \quad (define \; found \; \#f)
\\& \quad (map \; (range \; (length \; expr)) \; (lambda \; (i)
\\& \qquad (define \; read \; (get \; expr \; i))
\\& \qquad // \; \dots
\\& \quad )))
\end{align*}
\end{figure}

In order to parse out the parenthetical, however, we will need an additional variable. 
This variable will aid us in parsing nested parentheses to separate out the top-level parenthetical.

We will need to handle three obvious classes of characters in our parsing of the parentheses:

\begin{itemize}
  \item An opening parenthesis.
  \item A closing parenthesis.
  \item Any other character.
\end{itemize}

Additionally, the class of a character may be disregarded if we have already
parsed a top-level parenthetical. Its parsing will be handled when we are ready
to recurse. siven these additions of case-handling, we insert $if \dots else$
statements as in Figure~\ref{fig:imperParseApproach}.

\begin{figure}[htp]
\footnotesize
\caption{}\label{fig:imperParseApproach}
\begin{align*}
& (define \; parse \; (lambda \; (expr) \; 
\\& \quad (define \; before)
\\& \quad (define \; accum \; nil)
\\& \quad (define \; paren)
\\& \quad (define \; found \; \#f)
\\& \quad (define \; nested \; 0)
\\& \quad (map \; (range \; (length \; expr)) \; (lambda \; (i)
\\& \qquad (define \; read \; (get \; expr \; i))
\\& \qquad (if \; (and \; (equal? \; '< \; read) \; (not \; found))
\\& \qquad \quad (\dots"1. \; an \; opening \; parenthesis"\dots)
\\& \qquad \quad (if \; (and \; (equal? \; '> \; read) \; (not \; found))
\\& \qquad \qquad (\dots"2. \; a \; closing \; parenthesis"\dots)
\\& \qquad \qquad (\dots"3. \; any \; other \; character"\dots))))))
\end{align*}
\end{figure}

Of course, we will need to combine any separated out parenthetical with the components 
occurring before and after it to form the designated response. Hence we provide the 
following $return$ statement in Figure~\ref{fig:postscriptReturn}.

\begin{figure}[htp]
\footnotesize
\caption{}\label{fig:postscriptReturn}
\begin{align*}
& (define \; parse \; (lambda \; (expr) \; 
\\& \quad \dots"variables"\dots
\\& \quad (map \; (range \; (length \; expr)) \; (lambda \; (i)
\\& \qquad (\dots"parse"\dots))
\\& \quad (if \; paren
\\& \qquad (concat \; (push \; before \; paren) \; (parse \; accum))
\\& \qquad expr)))
\end{align*}
\end{figure}

Now we implement our nesting logic and the final algorithm. Nesting will be handled 
based on one of the following occurrences.

\begin{itemize}
  \item A once nested expression was just opened.
  \item An expression was just closed to be un-nested.
\end{itemize}
 Parentheses occurred within a nested expression.

The first and second cases are handled under the conditionals for their respective 
character classes, and in either class under another nesting case, the third will be 
handled.

The last components missing from our implementation are the building up of an 
accumulator and the setting of the various components to the accumulator. We will 
implement these portions in the code of Figure~\ref{fig:fullImperParser}.

\begin{itemize}
  \item When the parenthetical is closed, it is recursively $parsed$ and set to the $paren$ variable.
  \item When a parenthetical is open, $before$ receives the accumulator value.
\end{itemize}

\begin{figure}[htp]
\footnotesize
\caption{}\label{fig:fullImperParser}
\begin{align*}
& (define \; parse \; (lambda \; (expr) \; 
\\& \quad (define \; before)
\\& \quad (define \; accum \; nil)
\\& \quad (define \; paren)
\\& \quad (define \; found \; \#f)
\\& \quad (define \; nested \; 0)
\\& \quad (map \; (range \; (length \; expr)) \; (lambda \; (i)
\\& \qquad (define \; read \; (get \; expr \; i))
\\& \qquad (if \; (and \; (equal? \; '< \; read) \; (not \; found))
\\& \qquad \quad ((set! \; nested \; (+ \; 1 \; nested))
\\& \qquad \quad \; (if \; (equal? \; nested \; 1)
\\& \qquad \qquad \; ((set! \; before \; accum)
\\& \qquad \qquad \quad (set! \; accum \; nil))
\\& \qquad \qquad \; (set \; accum \; (push \; accum \; read)))
\\& \qquad \quad (if \; (and \; (equal? \; '> \; read) \; (not \; found))
\\& \qquad \qquad ((set! \; nested \; (- \; 1 \; nested))
\\& \qquad \qquad \; (if \; (equal? \; nested \; 0)
\\& \qquad \qquad \quad \; ((set! \; found \; \#t)
\\& \qquad \qquad \qquad (set! \; paren \; (parse \; accum))
\\& \qquad \qquad \qquad (set! \; accum \; nil))
\\& \qquad \qquad \quad \; (set \; accum \; (push \; accum \; read)))
\\& \qquad \qquad (set \; accum \; (push \; accum \; read)))))
\\& \quad (if \; paren
\\& \qquad (concat \; (push \; before \; paren) \; (parse \; accum))
\\& \qquad expr)))
\end{align*}
\end{figure}

\subsection{From Imperative to Functional}
From the above final implementation of our program we can derive a functional 
version. The differences will be based on the following principles of functional 
programming:

\begin{itemize}
  \item Values shall not be mutated.
  \item Control-flow shall not be explicit.
  \item Recursion is a fundamental idea.
\end{itemize}

Let's begin by abiding to the second rule, inspired by the third. The first thing 
you will notice is that all variables were made function arguments. This is because 
in a pure function, the only state is provided by the arguments. Hence when 
recursing, we will need to pass all required data to the function as argument.

Also of note is the fact that rather than maintain an index of the list on which we 
are operating, we pass as argument to the recursive call only subsequent characters, 
i.e., those which have yet to be read. This is both logical in that our progress in 
navigating the list is maintained, and idiomatic as you have seen in prior programs 
written in our Symbolic Language.

\begin{figure}[htp]
\footnotesize
\caption{}\label{fig:funParse}
\begin{align*}
& (define \; funparse \; (lambda \; 
\\& \quad (expr \; nested \; before \; paren \; accum \; found) \; 
\\& \quad (if \; (null? \; expr)
\\& \qquad (if \; (not \; (null? \; paren))
\\& \qquad \quad (concat \; (push \; before \; paren) \; (funparse\_ \; accum))
\\& \qquad \quad expr)
\\& \qquad ((let \; read \; (get \; expr \; 0)
\\& \qquad \quad \; (if \; (and \; (equal? \; read \; '<) \; (not \; found)))
\\& \qquad \qquad \; ((set! \; nested \; (+ \; 1 \; nested))
\\& \qquad \qquad \quad (if \; (equal? \; 1 \; nested)
\\& \qquad \qquad \qquad ((set! \; before \; accum)
\\& \qquad \qquad \qquad \; (set! \; accum \; nil))
\\& \qquad \qquad \qquad (set \; accum \; (push \; accum \; read))))
\\& \qquad \qquad \quad (if \; (and \; (equal? \; read \; '>) \; (not \; found)))
\\& \qquad \qquad \quad \; ((set! \; nested \; (- \; 1 \; nested))
\\& \qquad \qquad \qquad (if \; (equal? \; 0 \; nested)
\\& \qquad \qquad \qquad \quad ((set! \; paren \; (funparse\_ \; accum))
\\& \qquad \qquad \qquad \quad \; (set! \; found \; \#t)
\\& \qquad \qquad \quad \; (set! \; accum \; nil))
\\& \qquad \qquad \qquad \quad (set \; accum \; (push \; accum \; read))))
\\& \qquad \quad \; (set \; accum \; (push \; accum \; read)))
\\& \qquad \; (funparse \; (cdr \; expr) \; nested \; before \; paren \; accum \; found)))))
\\& (define \; funparse\_ \; (lambda \; (expr)
\\& \quad (funparse \; expr \; 0 \; '() \; '() \; '() \; \#f))) \; 
\end{align*}
\end{figure}

The final portion of our program includes a definition of $funparse\_$. This was merely 
for convenience, as $funparse\_$ provides all of the initialization values as argument 
to $funparse$.

We now remove mutation to achieve implementation of the final principle we listed. Our 
means of achieving this is by allowing all values to be function arguments or expressions 
operating on arguments.

\begin{figure}[htp]
\footnotesize
\caption{}\label{fig:immutableFunparse}
\begin{align*}
& (define \; funparse \; (lambda \; (expr \; nested \; before \; paren \; accum \; found) \; 
\\& \quad (if \; (null? \; expr)
\\& \qquad (if \; (not \; (null? \; paren))
\\& \qquad \quad (concat \; (push \; before \; paren) \; (funparse\_ \; accum))
\\& \qquad \quad expr)
\\& \qquad (if \; (and \; (equal? \; '< \; (get \; expr \; 0)) \; (not \; found))
\\& \qquad \quad (if \; (equal? \; nested \; 0)
\\& \qquad \qquad (funparse \; 
\\& \qquad \quad (cdr \; expr) \; (+ \; nested \; 1) \; 
\\& \qquad \quad accum \; paren \; 
\\& \qquad \quad nil \; found)
\\& \qquad \qquad (funparse \; 
\\& \qquad \quad (cdr \; expr) \; (+ \; nested \; 1) \; 
\\& \qquad \quad before \; paren \; 
\\& \qquad \quad (push \; accum \; (car \; expr)) \; found))
\\& \qquad \quad (if \; (and \; (equal? \; '> \; (get \; expr \; 0)) \; (not \; found))
\\& \qquad \qquad (if \; (equal? \; nested \; 1)
\\& \qquad \qquad \quad (funparse \; 
\\& \qquad \qquad (cdr \; expr) \; (- \; nested \; 1) \; 
\\& \qquad \qquad before \; (funparse\_ \; accum) \; 
\\& \qquad \qquad nil \; \#t)
\\& \qquad \qquad \quad (funparse \; 
\\& \qquad \qquad (cdr \; expr) \; (- \; nested \; 1) \; 
\\& \qquad \qquad before \; paren \; 
\\& \qquad \qquad (push \; accum \; (car \; expr)) \; found))
\\& \qquad \qquad (funparse \; 
\\& \qquad \quad (cdr \; expr) \; nested \; 
\\& \qquad \quad before \; paren \; 
\\& \qquad \quad (push \; accum \; (car \; expr)) \; found))))))
\\& (define \; funparse\_ \; (lambda \; (expr)
\\& \quad (funparse \; expr \; 0 \; '() \; '() \; '() \; \#f))) \; 
\end{align*}
\end{figure}

You should begin to see how our rewrite of this algorithm reads much more as an inductive 
definition than as a description of a process. In the following section we will make this 
even more evident.

\subsection{Adopting a Few Conventions}
There are a few vestiges of our initial, imperative implementation which we will now remove. 
Of note is the prior $define$ keyword that was appropriately substituted by $letrec$, with 
$funparse\_$ then being another definition within the $letrec$ procedure.

\begin{figure}[htp]
\footnotesize
\caption{}\label{fig:letrecFunparse}
\begin{align*}
& (letrec \; funparse \; (lambda \; (funparse \; expr \; nested \; before \; paren \; accum \; found) \; 
\\& \quad (let \; 
\\& \qquad funparse\_ \; 
\\& \qquad (lambda \; (expr) \; (funparse \; expr \; 0 \; '() \; '() \; '() \; \#f))
\\& \qquad (if \; (null? \; expr)
\\& \qquad \quad (if \; (null? \; paren) \; 
\\& \qquad \qquad expr
\\& \qquad \qquad (concat \; (push \; before \; paren) \; (funparse\_ \; accum)))
\\& \qquad \quad (if \; (and \; (equal? \; '< \; (car \; expr)) \; (not \; found))
\\& \qquad \qquad (if \; (equal? \; nested \; 0)
\\& \qquad \qquad \quad (funparse \; 
\\& \qquad \qquad (cdr \; expr) \; (+ \; nested \; 1) \; 
\\& \qquad \qquad accum \; paren \; 
\\& \qquad \qquad nil \; found)
\\& \qquad \qquad \quad (funparse \; 
\\& \qquad \qquad (cdr \; expr) \; (+ \; nested \; 1) \; 
\\& \qquad \qquad before \; paren \; 
\\& \qquad \qquad (push \; accum \; (car \; expr)) \; found))
\\& \qquad \qquad (if \; (and \; (equal? \; '> \; (car \; expr)) \; (not \; found))
\\& \qquad \qquad \quad (if \; (equal? \; nested \; 1)
\\& \qquad \qquad \qquad (funparse \; 
\\& \qquad \qquad \quad (cdr \; expr) \; (- \; nested \; 1) \; 
\\& \qquad \qquad \quad before \; (funparse\_ \; accum) \; 
\\& \qquad \qquad \quad nil \; \#t)
\\& \qquad \qquad \qquad (funparse \; 
\\& \qquad \qquad \quad (cdr \; expr) \; (- \; nested \; 1) \; 
\\& \qquad \qquad \quad before \; paren \; 
\\& \qquad \qquad \quad (push \; accum \; (car \; expr)) \; found))
\\& \qquad \qquad \quad (funparse \; 
\\& \qquad \qquad (cdr \; expr) \; nested \; 
\\& \qquad \qquad before \; paren \; 
\\& \qquad \qquad (push \; accum \; (car \; expr)) \; found)))))) \; \dots)
\end{align*}
\end{figure}

\subsection{The Parser}
The parser now works as in Figure~\ref{fig:parserExample}. 

\begin{figure}[htp]
\footnotesize
\caption{}\label{fig:parserExample}
\begin{align*}
& (letrec \; parse \; (lambda \; (\dots) \; \dots)
\\& \quad (parse \; '(< \; a \; > \; < \; b \; < \; c \; > \; > \; < \; d \; >)))
\\& \implies \; ((a) \; (b \; (c)) \; (d))
\end{align*}
\end{figure}

\subsection{Evaluation}
Combining the prior evaluator with the new addition of the parser, we have the behavior you would have expected.
