
\chapter{Designing Primitive Procedures}
\section{Introduction}
We have formed a language constituent of Symbolic Expressions. Additionally, we 
have an array of useful and primitive Lambda Calculus functions at our disposal. 
Now in order to build expressive and powerful programs, it will be helpful to 
define a library of useful Symbolic Functions for manipulation of the various 
data-types we have formed by abstraction.

\section{Predicates}
We begin with a trivial example, a boolean inverter. Our definition is listed as 
an S-Expression because when we are ready to make use of it, we will include that 
pair in a call to `let*`. The name of the function is the first element, because 
this is the variable to which it will be assigned. The actual function definition 
is very familiar. We form a lambda of a single function that results in 
conditional behavior; if `x` is true, it results in `f`, but if `x` is false, it 
results in `t`.

\begin{align*}
& (not \; (\text{lambda} \; (x) \; (if \; x \; \#f \; \#t)))
\end{align*}

Now we add to our current assortment of numeric predicates the functions `<` and 
`>`. These predicates are lambdas accepting two values, that they may be compared. 
With less-than-equal-to already defined as `leq`, both of these relations are 
trivial to define.

\begin{align*}
& (< \; &(\text{lambda} \; (x \; y) \; (and \; (leq \; x \; y) \; (not \; (eq \; x \; y)))))
\\& (> \; &(\text{lambda} \; (x \; y) \; (not \; (leq \; x \; y))))
\end{align*}

An important conveniency to note in the above forms, is their nature given their 
Lambda Calculus definitions. That is, since they compile down to a curried form of 
a function, in other words, a function returning a function, they are very nice to 
work with. Let's look at an example form and apply appropriate reductions to get a 
better view of this function's nature.

\begin{align*}
& (> \; 2)
\\& \implies \; &((\text{lambda} \; (x \; y) \; (not \; (leq \; x \; y))) \; 2)
\\& \implies \; &(\text{lambda} \; (y) \; (not \; (leq \; 2 \; y))) \; 
\end{align*}

What we see is that when a single value is passed, we achieve a convenient 
function ready to compare in terms of that parameter. This may seem obvious, 
familiar, or redundant, but the convenience of this fact should not be taken for 
granted. This phenomenon is known as currying; it can prove very useful in 
providing clearness to your expressions; our look at `>` was only an example of 
what is going on in all of the functions we discuss.

This ability to supply the arguments of a function one at a time makes for very 
legible code. Below is an example of an inductive definition utilizing this 
functionality. The predicate is `<` with the first argument supplied as two, the 
step is the `pred` function, and the combinator is multiplication.

Generalized Inductive Calculator
\begin{align*}
& (\text{letrec} \; 
\\& \quad induct \; 
\\& \quad (\text{lambda} \; 
\\& \qquad (induct \; pred \; num \; step \; combo) \; 
\\& \qquad (if \; 
\\& \qquad \quad (pred \; num) \; 
\\& \qquad \quad num \; 
\\& \qquad \quad (combo \; num \; (induct \; pred \; (step \; num) \; step \; combo)))) \; 
\\& \quad \; (induct \; (> \; 2) \; 6 \; pred \; *))
\\& \implies \; 720
\end{align*}

This reads very well, as "perform induction while greater than 2 from 6 by means 
of decrement combining with multiplication", i.e., find the factorial of 6. Let's 
look at another instance of this, taking advantage of the interchangeability of 
the definition, and once again of partial function application. This time we 
induce addition up to and at five, stepping by two in each step up from one.

\begin{align*}
& (\text{letrec} \; 
\\& \quad \; induct \; 
\\& \quad \; \dots
\\& \qquad (induct \; (< \; 5) \; 1 \; (+ \; 2) \; +))
\\& \implies \; 16
\end{align*}

Our answer coincides with what you would expect, the sum of odd numbers one 
through seven. Hopefully you feel that our language displays complexity well. The 
above example handled a very generic problem type elegantly and concisely. It is 
thanks to our adding of abstraction as we go that we are able to make these 
creations both clear and versatile, and ones in which the creator can take pride.

Lastly we provide predicates to determine whether a given number is even or odd. 
This function is once again very easy given our convenient Lambda Calculus 
primitives. The following two functions simply check for a specific value of the 
modulus division by two applied to a number; this is the essence of parity.

\begin{align*}
& (odd? \; &(\text{lambda} \; (x) \; (eq \; (mod \; x \; 2) \; 1)))
\\& (even? \; &(\text{lambda} \; (x) \; (eq \; (mod \; x \; 2) \; 0)))
\end{align*}

Use cases for these two predicates will arise later, but for now they are good at 
their simple duties, determining parity.

\section{Higher-Order Functions}
In writing clear and concise expressions, it is often useful to have at your 
disposal higher-order functions (*HOF*), that is, functions that (a) return 
functions, (b) accept functions as arguments, or (c) do both *a* and *b*.

Hopefully you caught something odd in what I just said, the fact that everything, 
hence any possible argument or resultant value, is a function! However since we 
have defined some primitive data-types, I am referring to non-data-symbolizing 
functions. This may seem a fine line, but you will often see this terminology 
tossed around, so you may as well utilize it even in when in the purest of 
functional languages.

We begin with a couple of type (c) HOFs. The first of the following serves to flip 
the argument ordering of a given function, and the second composes two functions.
\begin{align*}
& (flip \; &(\text{lambda} \; (func \; a \; b) \; (func \; b \; a)))
\\& (compose \; &(\text{lambda} \; (f \; g) \; (\text{lambda} \; (arg) \; (f \; (g \; arg)))))
\end{align*}

The function `flip` is very convenient when aiming to apply only the second 
argument of a function, leaving the other free. The design of `flip` is quite 
simple, it merely accepts a function, then two arguments, and returns application 
of them in reverse order. Despite the simplicity of its operation, it can very 
greatly reduce the complexity of an expression. As an example, look at the 
following definition of a singleton constructor, that is, a creator of a pair with 
a single element.

Singleton Constructor
\begin{align*}
& ((flip \; cons) \; nil)
\end{align*}


`compose` allows the results of various manipulations to be piped from one to 
another. A beautiful example of this is a linear function creator. Below is the 
function accepting slope and y-intercept as its two arguments.

Linear Function Generator
\begin{align*}
& (\text{lambda} \; 
\\& \quad \; (m \; b) \; 
\\& \quad \; (compose \; 
\\& \qquad \; (+ \; b) \; 
\\& \qquad \; (* \; m)))
\end{align*}

One of the most important HOFs is defined next. `fold` serves to accumulate a list 
of values into a single resultant value, based on a function of combination and a 
starting value. Note that this function is recursive and will be provided using 
`letrec`. Other functions accepting their name as the first argument should be 
assumed to follow the same practice.

\begin{align*}
& (fold \; (\text{lambda} \; (fold \; func \; accum \; lst)
\\& \quad (if \; (null? \; lst)
\\& \qquad accum
\\& \qquad (fold \; func \; (func \; accum \; (car \; lst)) \; (cdr \; lst)))))
\end{align*}

Our definition of `fold` is as a manipulator of a list returning an accumulated 
value at the end of a list, and at other points recursing with the `cdr` of the 
list and an accumulator as determined by the passed function. If the meaning of 
`fold` is still unclear to you, consider some of these examples.

\begin{align*}
& (fold \; + \; 0 \; '(1 \; 2 \; 3)) \; \implies \; 6
\\& (fold \; * \; 0 \; '(1 \; 2 \; 3 \; 4)) \; \implies \; 24
\end{align*}

As you can now see, the folding of an infix operation $a \bullet b$ over a 
sequence $a, b, c, ...$ is the nested application of the operation, or the 
following.

\begin{align*}
& ( \; \dots \; ((a \; \bullet \; b) \; \bullet \; c) \; \dots \; )
\end{align*}

As a complement to `fold` we define `reduce`. `reduce` is just like `fold` except 
right-associative; Hence the function applications are nested just like the `cons` 
basis of these lists.

\begin{align*}
& (reduce \; (\text{lambda} \; (reduce \; func \; end \; lst)
\\& \quad (if \; (null? \; lst)
\\& \qquad end
\\& \qquad (func \; (car \; lst) \; (reduce \; func \; end \; (cdr \; lst))))))
\end{align*}

Our definition of `reduce` is as a manipulator of a list returning an accumulated 
value at the end of a list, and at other points returning a manipulation of a 
recursion with the `cdr`, manipulated by the passed function. If we do an 
expansion of an infix operator for `reduce` as we did for `fold` we achieve 
something like the following when dealing with a list $... , x, y, z$

\begin{align*}
& ( \; \dots \; (x \; \bullet \; (y \; \bullet \; z)) \; \dots \; )
\end{align*}

Together `reduce` and `fold` are sufficient basis for any iterative process. Now 
we will provide an inverse operation for constructing a list given a construction 
criterion. `unfold` serves to invert a folding.

\begin{align*}
& (unfold \; (\text{lambda} \; (unfold \; func \; init \; pred)
\\& \quad (if \; (pred \; init)
\\& \qquad (cons \; init \; nil)
\\& \qquad (cons \; init \; (unfold \; func \; (func \; init) \; pred)))))
\end{align*}

To clarify the distinction between `fold` and `reduce`, we display the manner in 
which they can be thought of as opposites.

\begin{align*}
& (fold \; (flip \; cons) \; nil \; '(1 \; 2 \; 3)) \; &\implies \; '(1 \; 2 \; 3)
\\& (reduce \; cons \; nil \; '(1 \; 2 \; 3)) \; &\implies \; '(1 \; 2 \; 3)
\end{align*}

This examples drives home that the difference between the two is in direction of 
association, `reduce` is the natural operation for right associative operations 
and `fold` for left associative operations.

Together our definitions of `fold` and `reduce` are sufficient for definition of 
any iterative process. `unfold` in addition provides us with a means of 
constructing arbitrary lists based on constructing rules. We will now implement a 
variety of derived iterative forms based on `fold` and `reduce`.

\section{Reductive Forms}
We begin with some extensions to our basic binary operators of arithmetic and 
boolean algebra. The structure of these definitions is similar to that of our 
early definitions of arithmetic, an iterative process on a base value; however, in 
this case the conditions and multitude of application are determined by a provided 
list.

All of the following forms, which we will refer to as *Reductive Forms* are 
dependent on `fold`. `fold` provides the generic versatile power to combine a list 
in an arbitrary way; hence you will see a variety of operations used in folding, 
so you may want to think back to the examples of the nested operator.

We begin with some definitions of arithmetic and boolean manipulations. The 
definitions of these forms are intuitive, each with an infix operator which fits 
the role very intuitively.

\begin{align*}
& (sum \; (\text{lambda} \; (lst) \; &(fold \; + \; 0 \; lst))
\\& (product \; (\text{lambda} \; (lst) \; &(fold \; * \; 1 \; lst))
\\& (and* \; (\text{lambda} \; (lst) \; &(fold \; and \; \#t \; lst))
\\& (or* \; (\text{lambda} \; (lst) \; &(fold \; or \; \#f \; lst))
\end{align*}

Now we expand our application field in defining some optimization functions, `min` 
and `max`. Both of these works by comparing each element with a running extreme 
value, swapping if a new extreme is found. The definition of `max` follows, a 
simple folding onto the higher value.

\begin{align*}
& (max \; 
\\& \quad (list)
\\& \quad (fold \; 
\\& \qquad (\text{lambda} \; 
\\& \qquad (old \; new)
\\& \qquad (if
\\& \qquad \qquad (> \; old \; new)
\\& \qquad \qquad old
\\& \qquad \qquad new))
\\& \quad (car \; list)
\\& \quad (cdr \; list)))
\end{align*}

The application of this function to $'(1 5 3 4)$, for example, would return 5. Our 
implementation of `min` is nearly identical, simply changing the criterion of the 
fold.

\begin{align*}
& (min \; 
\\& \quad (list)
\\& \quad (fold \; 
\\& \qquad (\text{lambda} \; 
\\& \qquad (old \; new)
\\& \qquad (if
\\& \qquad \qquad (> \; old \; new)
\\& \qquad \qquad old
\\& \qquad \qquad new))
\\& \quad (car \; list)
\\& \quad (cdr \; list)))
\end{align*}

Next we define some methods that aid in treatment of lists in their entirety, 
`length` and `reverse`. `length` is one of the simplest folds you could define, 
folding by increment. `reverse` on the other hand, is not as obvious in its means 
of operation; it folds by means of a swap operation, $(flip cons)$, in this way 
forming a fully reversed list.

\begin{align*}
& (length \; &(\text{lambda} \; (lst) \; (fold \; (\text{lambda} \; (x \; y) \; (+ \; x \; 1)) \; 0 \; lst)))
\\& (reverse \; &(\text{lambda} \; (lst) \; (fold \; (flip \; cons) \; nil \; lst)))
\end{align*}

Now we provide a special function for determining associations in a list meant as 
a table. The setup of these lists is like the following, where each element is a 
list, with the first element serving as a key, and the second serving as a value.

\begin{align*}
& '((apple \; &red)
\\& \quad (pear \; &green)
\\& \quad (banana \; &yellow))
\end{align*}

In determining the association, we `fold` with the aim of reaching a value with a 
key matching that for which we are searching.

\begin{align*}
& (assoc \; (\text{lambda} \; (x \; list)
\\& \quad \; (fold \; 
\\& \qquad (\text{lambda} \; 
\\& \qquad (accum \; item) \; 
\\& \qquad (if \; 
\\& \qquad \qquad (equal? \; item \; (car \; x))
\\& \qquad \qquad (cdr \; x)
\\& \qquad \qquad accum)))
\\& \qquad \#f
\\& \qquad list)))
\end{align*}

`assoc` is very important in modeling hash-tables, and in general keeping track of 
named values. If `assoc` were applied to the table displayed prior with `banana` 
as a key, it would evaluate to `yellow`. Here is the full form, with `table` 
referring to the aforementioned table.

\begin{align*}
& (assoc \; 'banana \; table) \; \implies \; 'yellow
\end{align*}

\section{List Manipulations}
Before we delve too far into manipulation of lists, we will define a very helpful 
list constructor as follows.

\begin{align*}
& (list \; 
\\& \quad (\text{lambda} \; (list \; a) \; 
\\& \quad (if
\\& \qquad (null? \; a)
\\& \qquad nil
\\& \qquad (\text{lambda} \; (rest)
\\& \qquad (cons \; a \; (list \; rest)))))
\end{align*}

Usage of `list` is very intuitive. To construct a list, pass each element as 
argument to the `list` function, ending with `nil`. The following is an example of 
usage.   

\begin{align*}
& (list \; 1 \; 2 \; 3 \; nil) \; &\implies \; '(1 \; 2 \; 3)
\end{align*}

In manipulating a list, there are two basic classes of operations, (a) mapping a 
list to a value, and (b) converting one list to another. We have thoroughly 
covered the former, starting first with general forms and then implementing some 
useful examples. Now we will move on to the latter.

In mapping one list to another, we will provide two generic functions. The first, 
`map`, will apply a single function to each element of a list; the second will 
filter out items based on a predicate. These functions are very useful, imagine 
for example finding a sum of squares or constructing a list of primes.

\begin{align*}
& (map \; 
\\& \quad (\text{lambda} \; 
\\& \qquad (func \; lst)
\\& \qquad (reduce
\\& \qquad (\text{lambda} \; 
\\& \qquad \qquad (x \; y) \; 
\\& \qquad \qquad (cons \; (func \; x) \; y)) \; 
\\& \qquad nil
\\& \qquad lst)))
\end{align*}

Our map implementation works as a reduction with `cons`; if this were the extent 
of the function, the initial list would be returned. However, each element is 
passed through the provided function to result in a list with modified elements. 
`filter` takes advantage of the same aspect of `reduce`; however, in its 
definition it casts away values not matching a predicate.

\begin{align*}
& (filter \; 
\\& \quad (\text{lambda} \; 
\\& \qquad (pred \; lst)
\\& \qquad (reduce
\\& \qquad (\text{lambda} \; 
\\& \qquad \qquad (x \; y)
\\& \qquad \qquad (if
\\& \qquad \qquad (pred \; x)
\\& \qquad \qquad (cons \; x \; y)
\\& \qquad \qquad y))
\\& \qquad nil
\\& \qquad lst)))
\end{align*}

Let's look at some examples of `map` and `reduce`; the extent of their usefulness 
was alluded to earlier, but below are some examples to clarify their usage.

\begin{align*}
& (map \; &(* \; 2) \; &'(1 \; 2 \; 3)) \; &\implies \; '(2 \; 4 \; 6)
\\& (filter \; &odd? \; &'(1 \; 2 \; 3 \; 4 \; 5 \; 6)) \; &\implies \; '(1 \; 3 \; 5)
\end{align*}

The above were very clear in their meaning, as one would hope. Now that we have 
some strong ways of manipulating a list, we will move on to means of adding 
elements to a list. We provide some functions for appending to a list, either a 
single element of a list of elements, i.e., *concatenation*. These functions serve 
as nice complements to the two which were defined above, as they allow for 
expansion to supersets, and the earlier two allow only for constructing a subset.

\begin{align*}
& (push \; &(\text{lambda} \; (a \; b) \; (reverse \; (cons \; b \; (reverse \; a)))))
\\& (concat \; &(\text{lambda} \; (a \; b) \; (fold \; push \; a \; b)))
\end{align*}

These list manipulations will prove very useful, and given our prior functions, 
were very concise and clear in definition. Below are some examples of `push` and 
`concat` applications.

\begin{align*}
& (push \; &4 \; &'(1 \; 2 \; 3)) \; &\implies \; '(1 \; 2 \; 3 \; 4)
\\& (concat \; &'(1 \; 2) \; &'(2 \; 4)) \; &\implies \; '(1 \; 2 \; 3 \; 4)
\end{align*}

\section{Conclusion}
We have amassed a variety of useful and versatile functions of symbolic 
expressions. With these in hand, we are ready to build complex and useful 
programs.
