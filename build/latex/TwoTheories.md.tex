
\chapter{Preface: A Tale of Two Theories}
\section{The Theory}
In the middle of the nineteen-thirties, two men were laying the foundations of 
computation. It was in the year of nineteen-thirty-six that each released his 
*magnum opus*. Alonzo Church revealed to the world a formal system of computation 
expressed by function definition and evaluation, and Alan Turing shared his vision 
of a hypothetical device, one which could simulate the logic of any algorithm. In 
retrospect, the two works provided equivalent foundation for this brave new world 
of computers.

\section{The Lambda Calculus}
The language developed by Church was one of very simple syntax. All expressions in 
the language were one of the following three types.

\begin{itemize}
  \item A function definition of the form $\lambda a b$ where $a$ is a variable and $b$ is an expression.
  \item A function evaluation of the form $(a)b$ where $a$ is a function and $b$ an expression.
  \item A variable reference.
\end{itemize}

This can be summarized using Backus-Naur Form as follows. An expression is said to 
be replaceable by one of the values separated by a vertical pipe; each of these 
can then in turn be reduced by referring again to the definition of `<expr>`.

\begin{align*}
& <expr> \; &::= \; &\lambda \; <var> \; <expr>
\\& \qquad \qquad \qquad &| \; &(<expr>)<expr>
\\& \qquad \qquad \qquad &| \; &<var>
\end{align*}

As far as the semantic meaning of this language, there is only one operation. Any 
expression of the second form can be reduced by substituting the passed value for 
instances of the argument in the function body.

The elegance and minimalism of this language is striking. Despite its appearance, 
this language is capable of universal computation; that is, this tiny language is 
sufficient for expression of any algorithm, i.e., any concept or thought. This 
language, however simple, can become tedious with the heavy nesting of expressions 
and very difficult to actually evaluate to something meaningful. A more practical 
outlook is necessary when dealing with implementation by a machine.

\section{Turing Machines}
Turing provided us with this different outlook, one of simplicity in 
implementation rather than expression. Turing presented a machine capable of 
shifting a tape, modifying values, and maintaining state, following a set of 
prescribed rules in its performance of these operations. A single rule is of the 
following form.

\begin{align*}
& <state> \; ::= \; A \; | \; B \; | \; \dots
\\& <value> \; ::= \; 0 \; | \; 1
\\& <direction> \; ::= \; R \; | \; L
\\& <rule> \; ::= \; <state> \; <value> \; <value> \; <direction> \; <state>
\end{align*}

For a given rule, the first two values, *state* and *value*, serve to identify 
whether a rule is applicable, and the following values describe the manipulations 
that should take place. A Turing ruleset would then be a sequence of rule 
expressions. A certain Turing Machine has a given ruleset to achieve a specific 
goal. Perhaps this goal is to find the roots of a polynomial, or add two bits. The 
bit adding example is very easily implemented in simple rules, but any more 
difficult task can become quite complex.

\section{Abstraction}
The goal of this book is to build upon these rudimentary definitions of 
computation to reach a level of abstraction at which the fruits of your 
programming become clear and expressive. We ascend from these mess-making 
foundations, exiting Turing's tarpit, with the aid of abstraction over 
fundamentals.

When you add abstraction, you are simply shedding details which make a concept 
hard to interpret in its entirety. Hence throughout this book, we will be shifting 
focus from Lambda Calculus functions, to a more symbolic language, then all the 
way back down to Turing Machines. Our journey will be all across the board, but 
the goal is for the reader to understand traversal between any level of 
abstraction, amongst which are:

\begin{itemize}
  \item Representing a Symbolic Language with Functional Primitives
  \item Simulating a Turing Machine in a Symbolic Language
  \item Interpreting a Functional Language
  \item Interpreting a Language in Itself
  \item Converting a Symbolic Language to Manipulations of a Register Machine
\end{itemize}

All of the above will be addressed directly, and other traversals will become 
apparent given this context.
